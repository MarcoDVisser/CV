\section{Marco D. Visser}\label{marco-d.-visser}

\begin{center}\rule{0.5\linewidth}{\linethickness}\end{center}

\emph{PhD candidate at Radboud University Nijmegen}. Visiting address:
Nijmegen, The Netherlands Heyendaalseweg 135.
\href{mailto:m.visser@science.ru.nl}{\nolinkurl{m.visser@science.ru.nl}}
-
\href{mailto:marco.d.visser@gmail.com}{\nolinkurl{marco.d.visser@gmail.com}}

\url{https://github.com/MarcoDVisser}

\begin{center}\rule{0.5\linewidth}{\linethickness}\end{center}

\subsubsection{Research experience}\label{research-experience}

\begin{description}
\tightlist
\item[Since 2011]
PhD candidate (Apr 2011 - present) at Institute for Water and Wetland
Research, Plant Ecology Group, Radboud University Nijmegen (The
Netherlands).

Predoctoral Fellow (Apr 2010 - present) at the Smithsonian Tropical
Research Institute, Gamboa (Panama).
\item[2009 - 2010]
Junior researcher (Sept 2009 - Feb 2010) at the Department of
Experimental Plant Ecology, Radboud University Nijmegen (The
Netherlands).
\item[2008-2009]
Short-term Fellow (Oct 2008 - Feb 2009) at the Smithsonian Tropical
Research Institute, Barro Colorado Island (Panama).

MSc. Thesis research (2008-2009) at the Smithsonian Tropical Research
Institute, Barro Colorado Island, Panama.

MSc. Thesis research (2008-2009) at the unit Mathematical and
Statistical Methods of Wageningen University.
\item[2007]
B.A. Thesis research (2007) at the Forest Research Institute Malaysia,
Pasoh Forest Reserve, Malaysia.
\item[2005]
Internship (2005) at the Forest Research Institute Malaysia, Kepong,
Malaysia.
\item[2004]
Internship (2004) at the Mammal Research Institute, Polish Academy of
Sciences, Bialowieza, Poland.
\item[2003]
Volunteer (2003) at the Mammal Research Institute, Polish Academy of
Sciences, Bialowieza, Poland.
\end{description}

\begin{center}\rule{0.5\linewidth}{\linethickness}\end{center}

\subsubsection{Education}\label{education}

\begin{description}
\tightlist
\item[September, 2009]
Wageningen University and research centre, M.Sc. (cum laude, highest
distinction at WU). Forestry and Nature Conservation, with a minor in
Mathematics and Statistical Methods.
\item[September, 2007]
Larenstein University of Applied Sciences, B.A. Forestry and Nature
Conservation, with specialization in Tropical Forestry.
\end{description}

\begin{center}\rule{0.5\linewidth}{\linethickness}\end{center}

\subsubsection{Publications (inc. submitted/ in
preparation)}\label{publications-inc.-submitted-in-preparation}

\begin{description}
\tightlist
\item[2011]
\textbf{1.} \textbf{M. D. Visser}, E. Jongejans, M. van Breugel, P. A.
Zuidema, Y. Chen, A. R. Kassim, H. de Kroon. 2011. Strict mast fruiting
for a tropical dipterocarp tree: A demographic cost-benefit analysis of
delayed reproduction and seed predation. Journal of Ecology. 99,
1033-1044.

\textbf{2.} \textbf{M. D. Visser}, S. Joseph Wright, Helene C.
Muller-Landau, Gemma Rutten and Patrick A. Jansen. Tri-trophic
interactions affect density dependence of seed fate in a tropical forest
palm. 2011, Ecology Letters. 14, 1093-1100.
\item[2012]
\textbf{3.} B. van Putten, \textbf{M. D. Visser}, P. A. Jansen and H. C.
Muller-Landau. Distorted- distance models for directional dispersal: a
general framework and its application to a wind-dispersed tropical
forest trees. Methods in Ecology and Evolution. 2012.

\textbf{4.} B. T. Hirsch, \textbf{M. D. Visser}, R. Kays and P. A.
Jansen. Quantifying seed dispersal kernels from truncated seed-tracking
data. Methods in Ecology and Evolution. 2012
\item[2013]
\textbf{5.} \textbf{M. D. Visser}. aprof: Amdahl's profiler, directed
optimization made easy. R package version 0.1 - 0.3.1.
http://cran.r-project.org/web/packages/aprof/index.html. 2013.
\item[2014]
\textbf{6.} P. A. Jansen, \textbf{M. D. Visser}, S. J. Wright, G.
Rutten, H. C. Muller-Landau. Negative density-dependence of seed
dispersal and seedling recruitment in a Neotropical palm. Ecology
Letters 17: 1111--1120. 2014.
\item[2015]
\textbf{7.} \textbf{M. D. Visser}, S. M. McMahon, C. Merow, P. M. Dixon,
S. Record and E. Jongejans. Speeding Up Ecological and Evolutionary
Computations in R; Essentials of High Performance Computing for
Biologists. PLoS Comput Biol 11(3): e1004140.
doi:10.1371/journal.pcbi.1004140. 2015.
\item[2016]
\textbf{8.} \textbf{M. D. Visser}, M. Bruijning, S. J. Wright, H. C.
Muller-Landau, E. Jongejans, L. S. Comita and H. de Kroon. Functional
traits as predictors of vital rates across the life-cycle of tropical
trees. Functional Ecology.
\item[In press]
\textbf{9.} M. Bruijning, \textbf{M. D. Visser}, H. C. Muller-Landau, S.
J. Wright, L. S. Comita, S. P. Hubbell, H. de Kroon, E. Jongejans.
Surviving in a cosexual world: a cost-benefit analysis of dioecy in
tropical trees. Major revision. American Naturalist.
\item[In revision]
\textbf{10.} E.J. Francis, H.C. Muller-Landau, S.J. Wright, \textbf{M.
D. Visser}, Y. Iida, A.R. Kassim, C. Fletcher, and S.P. Hubbell.
Re-evaluating the functional significance of wood density for
interspecific variation in growth and survival in tropical trees. Global
Ecology and Biogeography
\item[In review]
\textbf{11.} \textbf{M. D. Visser}, S. Joseph Wright, Helene C.
Muller-Landau, Eelke Jongejans, Liza S. Comita, Hans de Kroon and Stefan
Schnitzer. Differential effects of lianas on population growth rates of
tropical forest trees. In prep for Ecology Letters.
\item[In prep]
\textbf{12.} \textbf{M. D. Visser}, S. Joseph Wright, Helene C.
Muller-Landau, Gemma Rutten and Patrick A. Jansen. Constraints on the
performance of a common tropical palm: an integral projection model of
density dependence. In preparation for Ecology Letters.

\textbf{13.} \textbf{M. D. Visser}, Helene C. Muller-Landau, Eelke
Jongejans, Liza S. Comita, Hans de Kroon and S. Joseph Wright.
Explaining variation among tree species in liana infestation. In prep
for Ecology.

\textbf{14.} \textbf{M. D. Visser}, Helene C. Muller-Landau, Eelke
Jongejans, Liza S. Comita, Hans de Kroon and S. Joseph Wright. The
comparative demography of tropical trees. In prep for Ecology.
\end{description}

\begin{center}\rule{0.5\linewidth}{\linethickness}\end{center}

~ ~ ~

\subsubsection{About my research}\label{about-my-research}

\begin{description}
\tightlist
\item[2015]
\textbf{Salguero-Gómez, R} (2015). Demography to infinity and beyond!
Journal of Ecology blog.
\url{https://jecologyblog.wordpress.com/2015/04/09/demography-to-infinity-and-beyond/}

\textbf{Wang, I} (2015). Recommendation F1000 prime.
\url{http://f1000.com/prime/725405210}
\item[2011]
\textbf{Sugden AM} (2011) Science Editors' choice. Ecology. The Enemy of
My Enemy is my? Science 334:569.

\textbf{Sugden AM} (2011) Science Editors' choice. Ecology. Why trees
skip a year. Science 333:386

\textbf{Rees M} (2011) Editor's Choice: Volume 99, Issue 4 (July).
Journal of Ecology.

\textbf{King, B} (2011), The enemy of my enemy is my friend. Smithsonian
Tropical Research Institute News 1:2

\textbf{Ecological Society of America} - young plant population
ecologist of the month (October 2011). Featured work: M. D. Visser et
al, 2011, Ecology Letters.

\textbf{Kouwen M} (2011) Mastjaar overtreft jaarlijkse zaadzetting.
Bionieuws 13:6.
\end{description}

\begin{center}\rule{0.5\linewidth}{\linethickness}\end{center}

\subsubsection{Grants and awards}\label{grants-and-awards}

\begin{description}
\item[2016]
\begin{itemize}
\tightlist
\item
  \textbf{Grant}: Academy Ecology Fund. Royal Dutch Academy of Sciences
  (KNAW), Quantifying the effects of extreme years on tropical tree
  dynamics: capitalizing a rare El Niño occurrence (6k).
\end{itemize}
\item[2011]
\begin{itemize}
\tightlist
\item
  \textbf{Grant}: NWO-ALW, What maintains the diversity of tropical tree
  species? Unravelling the importance of niche and neutrality with a
  life cycle approach. Co-wrote with Hans de Kroon, Helene
  Muller-Landau, Eelke Jongejans, S. J. Wright, P.A. Zuidema, P.A.
  Jansen and S. Tuljapurkar (230k).
\end{itemize}
\item[2009]
\begin{itemize}
\tightlist
\item
  \textbf{Award}: WUF-KLV thesis prize for the best thesis in the life
  sciences from Wageningen University awarded for my MSc thesis:
  Density-dependent dispersal and seed predation in a Neotropical palm.
\end{itemize}
\item[2008]
\begin{itemize}
\tightlist
\item
  \textbf{Grant}: Smithsonian Tropical Research Institute, short term
  fellowship awarded for the study: Quantifying density-dependent
  responses of seed predators in the Neotropical palm Attalea butyracea.
  (\$ 5k).
\end{itemize}
\end{description}

\begin{center}\rule{0.5\linewidth}{\linethickness}\end{center}

\subsubsection{International
presentations}\label{international-presentations}

\begin{description}
\tightlist
\item[2015]
\textbf{Workshop} at the British Ecological Society Annual Meeting.
December 2015, Edinburgh. Speeding Up Ecological and Evolutionary
Computations in R; Essentials of High Performance Computing for
Biologists. Organizer.

\textbf{Workshop} at the Evolutionary Demography Society Annual Meeting.
October 2015, Lunteren. Speeding Up Ecological and Evolutionary
Computations in R; Essentials of High Performance Computing for
Biologists. Organizer.

\textbf{Speaker} at the at the Ecological Society of America Annual
Meeting 2015. August 2015, Baltimore. Differential effects of lianas on
population growth rates of tropical forest trees.

\textbf{Workshop} at the at the Ecological Society of America Annual
Meeting 2015. August 2015, Baltimore. Demography in a Continuous World:
New Advances in Integral Projection Models (IPMs). Co-organizer.

\textbf{Workshop} at the at the British Ecological Society Symposium
``Demography Beyond The Population''. March 2015, Sheffield. Speeding Up
Ecological and Evolutionary Computations in R; Essentials of High
Performance Computing for Biologists.

\textbf{Speaker} at the British Ecological Society Symposium
``Demography Beyond The Population''. March 2015, Sheffield.
Differential effects of lianas on population growth rates of tropical
forest trees.
\item[2014]
\textbf{Short Workshop} at the Yale School of Forestry \& Environmental
Studies. December 2014, New Haven. Speeding Up Ecological and
Evolutionary Computations in R; Essentials of High Performance Computing
for Biologists.
\item[2012:]
\textbf{Invited speaker} at the conference ``Everything disperses to
Miami'', December 14 - December 16, 2012, the University of Miami. The
fitness consequences of dispersal for a tropical palm; the role of
dispersers, natural enemies and negative density dependence.

\textbf{Invited speaker} at the Max Planck Intitute for Demographic
Research, workshop on Integral Projection Models, Rostock Germany. June
2012. A Blueprint for speeding-up calculations in R.

\textbf{Speaker} at the Netherlands Annual Ecology Meeting. February
2012. Quantifying dispersal kernels through inverse modeling.
\item[2010]
\textbf{Invited speaker} at the 5th International Symposium-Workshop on
Frugivores and Seed Dispersal. Montpellier, France. June 2010. Measuring
dispersal kernels through inverse modeling: density dependence of seed
dispersal in a Neotropical palm.

\textbf{Speaker} at Plant Population Biology: Crossing Borders.
Gfo-conference, Nijmegen, Netherlands. May 2010. Strict mast fruiting
for a tropical dipterocarp tree: a demographic cost-benefit analysis
\item[2009]
\textbf{Oral presentation} at the Smithsonian Tropical Research
Institute. Panama. December 2009. Density-dependent dispersal and seed
predation in a Neotropical palm.
\item[2008]
\textbf{Oral presentation} at the workshop on stochastic elasticity and
matrix modeling. Nijmegen, the Netherlands, June 2008. Strict masting in
the tropical tree species Shorea leprosula: demographic consequences and
evolutionary benefit of predator satiation.
\item[2007]
\textbf{Oral presentation} at the International workshop in Matrix
models of plant populations. Sogndal, Norway, June 2007. Demographic
consequences of strict masting for two tropical tree species Shorea
leprosula and Shorea parvifolia.
\end{description}

\begin{center}\rule{0.5\linewidth}{\linethickness}\end{center}

\subsubsection{Reviewer for scientific
journals}\label{reviewer-for-scientific-journals}

Biotropica, Canadian Journal of Forest Research, Ecology, Ecology and
Evolution, Ecology Letters, Journal of Biogeography, Journal of Ecology,
Methods in Ecology and Evolution, PLOS computational biology, The R
Journal.

\begin{center}\rule{0.5\linewidth}{\linethickness}\end{center}
